\documentclass{article}

\usepackage{graphics} 
\usepackage{graphicx}
\usepackage{epsfig} 
\usepackage{amsmath} 
\usepackage{amssymb}  
\usepackage{amsfonts}
\usepackage{amsmath} 
\usepackage{subfigure}

\begin{document}
Dynamics of a multirotor

\begin{align}
\dot{p} &= v \nonumber \\
m\dot{v} &= mge_3 + TRe_3 \nonumber \\
\dot{R} &= R\hat{\omega} \nonumber \\
J\dot{\omega} &= -\omega \times J\omega + \tau. \nonumber 
\end{align}

Define $X_1 = p$, $X_2 = v$, and errors
\begin{align}
e_p &= X_1 - p^r \nonumber \\
e_v &= X_2 - v^r. \nonumber
\end{align}

Error dynamics
\begin{align}
\dot{e}_p &= v - \dot{p}^r \nonumber \\
&= e_v \nonumber \\
\dot{e}_v &= ge_3 + \frac{T}{m}Re_3 - \dot{v}^r. \nonumber 
\end{align}

The collective thrust $T$ is designed as
\begin{align}
n_d &= m(-K_pe_p - K_ve_v -ge_3 + \dot{v}^r) \nonumber \\
R_de_3 &= \frac{n_d}{|n_d|}  \nonumber \\
T &= n_d^TRe_3 \nonumber \\
&= |n_d| (e_3^TR_d^TRe_3). \nonumber
\end{align}

The error dynamics for $e_v$ with the collective thrust
\begin{align}
\dot{e}_v &= ge_3 - \dot{v}^r + \frac{n_d}{m} +\frac{T}{m}Re_3 - \frac{n_d}{m}\nonumber \\
 &= ge_3 - \dot{v}^r  - K_pe_p - K_ve_v -ge_3 + \dot{v}^r +\frac{T}{m}Re_3 - \frac{n_d}{m} \nonumber  \\
&= -K_pe_p -K_ve_v  + \frac{1}{m}(n_d^TRe_3)Re_3 -\frac{n_d}{m} \nonumber \\
&= -K_pe_p-K_ve_v + \frac{|n_d|}{m}\{(e_3^TR_d^TRe_3)Re_3 - R_de_3\}. \nonumber
\end{align}

The attitude error function is defined as $\Psi = \frac{1}{2}\text{tr}(I_3 - R_d^TR)$, and the dynamics of the error function is 
\begin{align}
\Psi &= \frac{1}{2}\text{tr}(I_3 - R_d^TR) \nonumber \\
\dot{\Psi} &=  -\frac{1}{2}\text{tr}(R_d^TR\hat{e}_\omega) \nonumber \\
&=  -\frac{1}{4}\text{tr}(-\hat{e}_\omega R^T R_d + R_d^T R \hat{e}_\omega) \nonumber \\
&= -\frac{1}{4}\text{tr}(\hat{e}_\omega(R_d^TR - R^TR_d)) \nonumber \\
&= -\frac{1}{2}\text{tr}(\hat{e}_\omega \hat{e}_R) \nonumber \\
&= e_R^Te_\omega \nonumber 
\end{align}
where $e_R = \frac{1}{2} (R_d^T R-R^T R_d) ^\vee$ and $e_\omega = \omega - R^TR_d\omega_d = \omega - \omega_d^R$.

Actually, $e_R$ could be interpreted in the following way
\begin{align}
R^TR_d &= I_3 + \sin\theta \hat{K} + (1-\cos\theta)\hat{K}^2 \nonumber \\
\hat{e}_R &= \frac{1}{2} (I_3 - \sin\theta \hat{K} + (1-\cos\theta)\hat{K}^2 - I_3 - \sin\theta \hat{K} -(1-\cos\theta)\hat{K}^2) \nonumber \\
&= -\sin\theta\hat{K} \nonumber \\
e_R &= -\sin\theta K \nonumber
\end{align}

If the angular velocity is designed to be $\omega = -K_Re_R + \omega_d^R$, the error $e_\omega$ becomes as $e_\omega = -K_Re_R$. Also, the attitude error dynamics becomes as
\begin{equation}
\dot{\Psi} = -e_R^TK_Re_R. \nonumber
\end{equation}

The error dynamics $\dot{e}_R$ is derived as
\begin{align}
\dot{e}_R &= \frac{1}{2}(R_d^T R \hat{e}_\omega +\hat{e}_\omega R^T R_d)^\vee \nonumber \\
&= \frac{1}{2}(\text{tr}(R^TR_d)I_3 - R^TR_d)e_\omega \nonumber \\
&= C(R^TR_d)e_\omega \nonumber
\end{align}
where $C =  \frac{1}{2}(\text{tr}(R^TR_d)I_3 - R^TR_d)$, $\dot{R}_d = R_d\text{S}(\omega_d)$ and $\omega_d^R = R^TR_d\omega_d$. 
The matrix $C(R^TR_d)$ is further analyzed as follows:
\begin{align}
C(R^TR_d) &= \frac{1}{2}\{(1+2\cos\theta)I_3 - (I_3 + \sin\theta\hat{K} + (1-\cos\theta)\hat{K}^2)\} \nonumber \\
&= \cos\theta I_3 +\frac{1}{2}\{-\sin\theta\hat{K} - (1-\cos\theta)\hat{K}^2\} \nonumber \\
&= \cos\theta I_3 +\frac{1}{2}\{ \hat{e}_R -\frac{1-\cos\theta}{\sin^2\theta}\hat{e}_R^2\} \nonumber \\
&= \cos\theta I_3 +\frac{1}{2}\{ \hat{e}_R -\frac{1}{1+\cos\theta}\hat{e}_R^2\} \nonumber
\end{align}

In addition, the determinant $\text{det}(C(R^TR_d)) = 4\cos\theta(\cos\theta + 1)$ where $\theta$ is the rotation angle from $R_d$ to $R$.

Therefore, the error dynamics $\dot{e}_R$ could be further analyzed by setting $\omega = -K_Re_K + \omega_d^R$ as follows:
\begin{align}
\dot{e}_R &= C(R^TR_d)e_\omega \nonumber \\
&= -C(R^TR_d)K_Re_R \nonumber \\
    %&= -(2\cos\theta I_3 + \hat{e}_R + \frac{1}{1+\cos\theta}\hat{e}_R^2)K_Re_R \nonumber
    &= -(\cos\theta I_3 + \frac{1}{2}\{\hat{e}_R - \frac{1}{1+\cos\theta}\hat{e}_R^2\})K_Re_R \nonumber
\end{align}

If the Lyapunov candidate is set as $V_R = \frac{1}{2}e_R^Te_R$, the directional time derivative of the function becomes:
\begin{align}
    \dot{V}_R &= e_R^T\dot{e}_R \nonumber \\
    &= -e_R^T(\cos\theta I_3 + \frac{1}{2}\{\hat{e}_R - \frac{1}{1+\cos\theta}\hat{e}_R^2\})K_Re_R \nonumber \\
    &= -\cos\theta \cdot e_R^T K_Re_R \nonumber
\end{align}

Let me start Lyapunov stability analysis for the whole system.
\begin{equation}
    V = \frac{1}{2}\{e_p^TK_pe_p + e_v^Te_v + e_R^Te_R\}+ \alpha e_p^T e_v\nonumber
\end{equation}

\begin{align}
    \dot{V} &= e_p^T K_p\dot{e}_p + e_v^T \dot{e}_v + e_R^T \dot{e}_R + \alpha e_p^T \dot{e}_v + \alpha e_v^T \dot{e}_p \nonumber \\
    &= e_p^T K_pe_v + e_v^T \left\{ -K_pe_p -K_ve_v + \frac{|n_d|}{m}\left[ (e_3^TR_d^TRe_3)Re_3-R_de_3 \right] \right\} -\cos\theta\cdot e_R^T K_R e_R  \nonumber \\
    &\;\;\;\;+\alpha e_v^T e_v + \alpha e_p^T \left\{ -K_pe_p -K_ve_v + \frac{|n_d|}{m} \left[ (e_3^TR_d^TRe_3)Re_3-R_de_3 \right] \right\}\nonumber \\
    &< -e_v^T K_v e_v + s^2 \frac{|e_v||n_d|}{m} - c^2 e_R^TK_Re_R + \alpha e_v^T e_v -\alpha e_p^TK_p e_p -\alpha e_p^TK_ve_v +\alpha s^2\frac{|e_p||n_d|}{m} \nonumber \\
    &< -\alpha e_p^TK_pe_p -e_v^T(K_v-\alpha I_3)e_v - c^2 e_R^T K_R e_R - \alpha e_p^T K_v e_v + s^2\frac{|e_v| + \alpha|e_p|}{m}|n_d| \nonumber \nonumber \\
    &< -\alpha e_p^TK_pe_p -e_v^T(K_v-\alpha I_3)e_v - c^2 e_R^T K_R e_R - \alpha e_p^T K_v e_v \nonumber \\
    &\;\;\;\;+ s^2(|e_v| + \alpha|e_p|)(K_p|e_p| + K_v |e_v| + B) \nonumber \\ 
    &< -\alpha e_p^TK_pe_p -e_v^T(K_v-\alpha I_3)e_v - c^2 e_R^T K_R e_R - \alpha e_p^T K_v e_v \nonumber \\
    &\;\;\;\;+ \alpha s^2 k_p|e_p|^2 + s^2k_v|e_v|^2 + s^2(\alpha k_v + k_p)|e_p||e_v| + s^2(|e_v| + \alpha|e_p|)B \nonumber \\ 
    &< e_p^T(-\alpha K_p + \alpha s^2 k_p I_3)e_p + e_v^T(-K_v+(\alpha + s^2 k_v) I_3)e_v - c^2 e_R^T K_R e_R \nonumber \\
    &\;\;\;\;- \alpha e_p^T K_v e_v + s^2(\alpha k_v + k_p)(\frac{a_1}{2}|e_p|^2 + \frac{a_2}{2}|e_v|^2) + s^2B(\frac{a_3+a_5}{2} + \frac{a_4}{2}|e_v|^2 +\frac{\alpha a_6}{2}|e_p|^2) \nonumber 
\end{align}
where $c$ and $s$ are parameters satisfying $c^4 = 1 - e_R^T e_R$ and $s^4 = e_R^Te_R$. Also, $k_v$ and $k_p$ are maximum eigenvalues of $K_v$ and $K_p$, respectively. 
To satisfy Young's inequality, $a_1a_2 = 1$, $a_3a_4 = 1$, $a_5a_6 = 1$.

\newpage










































\end{document}
